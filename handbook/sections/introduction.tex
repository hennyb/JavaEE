\section{Introduction}
\textsc{Votes - Electronic Polls Made Easy} is the examination project of the JavaEE course in summer semester 2014 at the University of Koblenz-Landau.

The Votes!-System supports electronic polls. The system shall be used mainly by university bodies
such as the examination office (Prüfungsausschuss). Many decisions in such bodies are made by
polls. Since a common meeting is often hard to organize due to crowded schedules of participants,
a voting system accessible via the Web is desirable.

Any member of the university must be able to define and conduct a voting. This role shall be called
organizer. Participants, i.e. the people making a choice, need not be member of the university.
The system shall support different types of polls, such as yes-no, one of n options, m of n options.
Voters must be able to abstain from voting (vote void, ungültige Stimme oder Enthaltung). Depend-
ing on the decision mode (absolute majority relative majority, simple majority), void votes are counted
differently.

Participants of a voting get a token, e.g. a transaction number, via E-Mail. This token represents their
ballot paper (Wahlschein). The system must ensure anonymity. At no point in time, the person sub-
mitting a ballot must be connectable to her voting.

However, some administrative procedures require that the system keeps track of who has voted and
who didn’t participate. The organizer can decide if such tracking is required. Optionally, the voting
system can send reminder mails to voters who did not yet participate.
Once a voting has been started, nobody can change the options and the participant list any more.

After the voting deadline expires, the organizer can view the results. To ensure anonymity, the results
may not be shown when an identification of a participant is possible (e.g. when tracking is enabled,
and only one participant submitted her decision).

A graphical representation of the results, e.g. as a bar chart or pie chart, would be nice. For subse-
quent polls, an organizer should be able to record participant lists. The lists can easily be reused
when the same participants occur many times, a usual situation in university bodies.