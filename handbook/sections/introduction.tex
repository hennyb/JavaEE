\section{Introduction}
\textsc{Votes - Electronic Polls Made Easy} is the examination project of the JavaEE Web Applications course, summer tern 2014 at the University of Koblenz-Landau.
The Votes-System provides means to conduct arbitrary electronic polls comparable to the Doodle\footnote{\url{http://doodle.com/}} scheduling tool, but is not specialized for time schedules.

The main use case for Votes is to simplify poll based decision finding of university bodies such as the examination office.
Due to busy schedules of members, meetings are hard to organize, which results in an unwanted delay of important decisions.
Votes provides a solution for such polls accessible via the Web, which renders meetings unnecessary to conduct polls where no further discussion is needed.

Any member of the university can register with Votes as \textit{Organizer}.
Once signed in, an organizer can create polls with an arbitrary amount of \textit{Items} he or she wishes to be voted on.
Such items can be questions, which only demand a \textsc{Yes/No} decision.
Or items can consist of \textit{n} options a participant can select.
Additionally the organizer can specify a minimum number (1 or more) of options the participant has to select.

Participants can be invited to a poll by simply adding his or hers email-address to the list of participants.
When a poll is started by its organizer, each participant will receive an message containing a \textit{Token} string, which will act as ballot, and an URL, where to conduct the vote until the poll's expiration date.
Before a participant can submit a vote, the Votes-System requires a valid token.
Without such a token a participant cannot view the poll.

To ensure anonymity certain restrictions are made.
Although Votes stores who has voted on which poll, no link can be established with what a participant has voted for.
Additionally each poll requires at least three participants and no result can be displayed if less than tree have participated.
Moreover, no intermediate results can be displayed.
